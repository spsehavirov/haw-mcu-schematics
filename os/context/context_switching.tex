\documentclass[margin=2pt]{standalone}
\usepackage[table]{xcolor}
\usepackage[utf8]{inputenc}
\usepackage[T1]{fontenc}

\usepackage{lmodern}
\usepackage{helvet}

\usepackage{tikz}

\usepackage{amsmath}
\usepackage[czech]{babel}
\usepackage{csquotes}
\usepackage{xparse}

\renewcommand{\familydefault}{\sfdefault}

% Use \phantom to hide text for exams
\renewcommand{\phantom}{}

\iflanguage{czech}{
	\newcommand{\lSave}{Ulož PCB}
	\newcommand{\lLoad}{Načti PCB}
	\newcommand{\lCPU}{CPU}
	\newcommand{\lOS}{Operační systém}
	\newcommand{\lInterrupt}{Přerušení nebo systémové volání}
}


\usetikzlibrary{
	hobby,
	babel,
	intersections,
	spath3,
	shapes.arrows,
	shapes.geometric,
	shapes.symbols,
	fit,
	backgrounds, 
	calc,
	tikzmark,
	decorations.pathreplacing,
	angles,
	arrows.meta,
	quotes,
	positioning,
}

\definecolor{themeBlue}{RGB}{1, 103, 143}
\definecolor{themeOrange}{RGB}{221, 109, 16}
\definecolor{themeTeal}{RGB}{18, 54, 69}
\definecolor{themeGrey}{RGB}{120, 121, 124}
\definecolor{themeGreen}{HTML}{99ced3}

\begin{document}
	\begin{tikzpicture}[
		node distance=.2cm,
		element/.style={draw, rounded corners=4pt, minimum width=3.5cm, minimum height=1cm},
		arrow/.style={draw, single arrow, single arrow head extend=3pt},
		process 1 arrow/.style={-{Triangle[scale=.5]}, line width=4pt, themeOrange!90},
		process 2 arrow/.style={process 1 arrow, themeBlue!90},
		process idle/.style={themeGrey, dotted, semithick},
		line to node/.style={-latex, themeGrey!75, shorten <= 4pt, shorten >= 2pt},
		line from node/.style={-latex, themeGrey!75, shorten <= 2pt, shorten >= 4pt},
		connect nodes/.style={-{Latex[scale=.6]}, themeGrey!75},
		block/.style={minimum width=2.5cm, draw, font=\scriptsize}
	]
	
	\begin{scope}[name=p1]
		\draw[process 1 arrow] (0,0) coordinate (p1 run 1 start) -- ++(down:2cm) coordinate (p1 run 1 end);
		\draw[process idle, shorten <= 2pt, shorten >= 2pt] (p1 run 1 end) coordinate (p1 idle start) -- ++(down:4cm) coordinate (p1 idle end);
		\draw[process 1 arrow] (p1 idle end) coordinate (p1 run 2 start) -- ++(down:2cm) coordinate (p1 run 2 end);
	\end{scope}
	
	\begin{scope}[name=p2, xshift=10cm]
		\coordinate (o) at (0,0);
		\draw[process idle, shorten >= 2pt] (p1 run 1 start-|o) coordinate (p2 idle 1 start) -- (p1 run 1 end-|o) -- ++(down:.6) coordinate (p2 idle 1 end);
		\draw[process 2 arrow] (p2 idle 1 end) coordinate (p2 run 1 start) -- ($ (p1 idle end-|o) +(0,.6) $) coordinate (p2 run 1 end);
		\draw[process idle, shorten <= 2pt] (p2 run 1 end) coordinate (p2 idle 2 start) -- (p1 run 2 end-|o) coordinate (p2 idle 2 end);
	\end{scope}
	
	\begin{scope}[name=cpu]
		\node (pcb 0 save) at ($ (p1 run 1 end)!.5!(p1 run 1 end-|o) $) [block, fill=themeOrange!10] { \lSave\space \texttt{0} };
		\node (pcb 1 load) at (pcb 0 save|-p2 run 1 start) [block, fill=themeBlue!10] { \lLoad\space \texttt{1} };
	
		\node (pcb 1 save) at ($ (p2 run 1 end)!.5!(p2 run 1 end-|p1 idle start) $) [block, fill=themeBlue!10] { \lSave\space \texttt{1} };
		\node (pcb 0 load) at (pcb 1 save|-p1 run 2 start) [block, fill=themeOrange!10] { \lLoad\space \texttt{0} };
	
		\draw[line to node] (p1 run 1 end) -- (pcb 0 save);
		\draw[line from node] (pcb 1 load) -- (p2 run 1 start);
		\draw[connect nodes] (pcb 0 save) -- (pcb 1 load);
	
		\draw[line to node] (p2 run 1 end) -- (pcb 1 save);
		\draw[line from node] (pcb 0 load) -- (p1 run 2 start);
		\draw[connect nodes] (pcb 1 save) -- (pcb 0 load);
	\end{scope}
	
	\begin{scope}[name=labels]
		\draw (p1 run 1 start) ++(up:.5) node [font=\bfseries] (p1 label) {$ \text{P}_{\text{\texttt{0}}} $};
		\draw (p2 idle 1 start) ++(up:.5) node [font=\bfseries] (p2 label) {$ \text{P}_{\text{\texttt{1}}} $};
		\draw (p1 label-|pcb 0 save) node [font=\footnotesize\bfseries] (cpu label) {\lCPU};
		\node at ($ (pcb 1 load.south)!.5!(pcb 1 save.north) $) [themeTeal, font=\scriptsize] { \lInterrupt };
	\end{scope}
	
	\begin{scope}[on background layer]
		\node (os frame) [fit=(p1 label)(cpu label)(p2 label), fill=yellow!15, outer sep=0, inner sep=0] {};
		\node (os label) at (os frame.north) [above, font=\scriptsize] { \lOS };
	\end{scope}
		
	\end{tikzpicture}
\end{document}
