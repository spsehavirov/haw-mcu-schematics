\documentclass{standalone}

\usepackage[utf8]{inputenc}
\usepackage[T1]{fontenc}

\usepackage{tikz}
\usepackage{circuitikz}
\usetikzlibrary{calc}

\begin{document}
\begin{circuitikz}[
    european, 
    straight voltages,
    3-bit decoder/.style={dipchip, num pins=14, no topmark, hide numbers, draw only pins={1-7, 10-12}, external pins width=0}
]

    % Operační zesilovače
    \foreach \i [remember=\i as \lasti] in {0, ..., 6} {
        \draw (0,  -1.75*\i) node[op amp,yscale=-1, scale=0.75] (op\i) {\i};
        \draw (op\i.+) to[short] ++(-0.5, 0) coordinate (op_p\i);;
        \draw (op\i.-) to[short, -*] ++(-1.5, 0) coordinate (op_m\i);
        \ifnum\i=0
            % skip first
        \else
            \draw (op_m\i) to[R, l_=$R$] (op_m\lasti);
            \draw (op_p\i) to[short, -*] (op_p\lasti);
        \fi   
    }
    
    \draw ($(op0.out)!0.5!(op6.out)$) ++(1.75,0) node[3-bit decoder] (d) {Dekodér};
    \foreach \pin/\label in {1/$ in_7 $, 7/$ in_ 1 $, 12/$ MSB $, 10/$ LSB $} {
        \ifnum\pin>9
            \node [left, font=\tiny] at (d.bpin \pin) {\label};
        \else
            \node [right, font=\tiny] at (d.bpin \pin) {\label};
        \fi  
    }
    % Propojky na dekoder
    \foreach \i [count=\h from 1] in {0, ..., 2} {
        \draw (d.bpin \h) |- ++(-0.25*\h, 0) |- (op\i.out);
    }
    \foreach \i [count=\pin from 4, count=\o from 0] in {3, ..., 6} {
        \draw (d.bpin \pin) -- ($ (d.bpin \pin) +(-1, 0) -(-.25*\o, 0) $) |- (op\i.out);
        
    }

    % Vystupy
    \foreach \p in {10, ..., 12} {
        \draw (d.bpin \p) to[short, -o] ++(1, 0) coordinate (y\p);
    }
    \draw (y11) node [label={[label distance=1em,]right:\Large{{$Y$}}}] {};

    \draw (op_m6) to[R={$R/2$}] ++(0, -2) node[ground, label={[label distance=0em]right:{$0 V$}}] (GND) {} ;

    \draw (op_m0) to[R,l_=$R$] ++(0, 3) coordinate (p_ref) to[short, -o] ++(0, 0.0) node[above] {\large{$U_{ref}$}};

    \draw (op_p0|-p_ref) -- (op_p0);
    \draw (op_p0|-p_ref) to[short, -o] ++(0, 0.0) node[above] {\large{$U_{x}$}};
    
    \draw (op_m1) to[blue, R, l_=$R$, v^<=$U_{LSB}$] (op_m0);
    

\end{circuitikz}
\end{document}
