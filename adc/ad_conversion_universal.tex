% ==============================================================================
% @file        ad_conversion.tex
% @author      Tomáš Michalek
% @date        2025-06-06
% @version     1.0
%
% @copyright   Copyright (C) 2025 SPŠE Havířov
% @license     EUPL-1.2-or-later
% @SPDX-License-Identifier EUPL-1.2
%
% @description
% This TikZ diagram illustrates the complete process of Analog-to-Digital
% conversion (ADC) in sequential stages, including:
%
% 1. Analog input signal (continuous waveform)
% 2. Sampling (impulses and vertical lines at sample points)
% 3. Quantization (horizontal levels with decision thresholds)
% 4. Coding (binary representation of quantized levels)
% 5. Digital reconstruction (piecewise step approximation)
%
% All stages are conditionally visible via toggle macros (e.g. \AnalogSignalVisible),
% allowing step-through visualizations for didactic use.
%
% The diagram supports multilingual captions through babel's \iflanguage switch.
% To generate a different language version, define additional blocks:
%   \iflanguage{<language>}{...}{...}
% and provide translations for all relevant labels.
%
% @build
% To compile this file:
% > pdflatex ad_conversion.tex
%
% This standalone file produces a cropped PDF image suitable for inclusion in
% educational materials such as presentations or printed worksheets.
%
% @dependencies
% - TikZ
% - PGF libraries: calc, arrows.meta, shapes.arrows, backgrounds, intersections
% - etoolbox, xstring, xparse, amsmath
% - Czech localization (babel[czech])
% ==============================================================================

\documentclass[tikz, border=2pt]{standalone}
\usepackage[utf8]{inputenc}
\usepackage[T1]{fontenc}

\usepackage{tikz}
\usepackage{amsmath}
\usepackage[czech]{babel}
\usepackage{helvet}
\usepackage{csquotes}
\usepackage{xparse}
\usepackage{xstring}
\usepackage{etoolbox}
\usepackage{xfp}

\usetikzlibrary{
    babel,
    calc,
    arrows.meta,
}

%% VISIBILITY CONFIGURATION

% Analog signal
\newcommand{\AnalogSignalVisible}{true}
\newcommand{\UseGenericVoltageLabel}{~true}

% Sampling
\newcommand{\SamplingImpulsesVisible}{~true}

\newcommand{\SampleIntervalsVisible}{~true}
\newcommand{\SampleLineVisible}{~true}
\newcommand{\SamplePointsVisible}{~true}
\newcommand{\SamplePointValueVisible}{~true}

% Quantization
\newcommand{\QuantizationLevelsVisible}{~true}
\newcommand{\QuantizationSubLevelsVisible}{~true}
\newcommand{\QuantizationLevelLabelsVisible}{~true}
\newcommand{\QuantizationPointsVisible}{~true}
\newcommand{\QuantizationPointValueVisible}{~true}
\newcommand{\QuantizationPointLineVisible}{~true}

% Coding
\newcommand{\CodingVisible}{~true}

% Reconstructed signal
\newcommand{\ReconstructionVisible}{~true}


% Use \phantom to hide text for exams
%\renewcommand{\phantom}{}

\iflanguage{czech}{
    \newcommand{\lTime}{t}
    \newcommand{\lVoltage}{U}
    \newcommand{\lVoltageInputSuffix}{vst}
    \newcommand{\lVoltageSamplingSuffix}{vz}
    \newcommand{\lQuantizationLevel}{Kvantizační úroveň}
    \newcommand{\lQuantizationDecisionLevel}{Rozhodovací úroveň}
}


\renewcommand{\familydefault}{\sfdefault}

\newcommand{\textsized}[2]{%
    {\fontsize{#1}{#1}\selectfont\strut #2}%
}

\definecolor{themeBlue}{RGB}{1, 103, 143}
\definecolor{themeOrange}{RGB}{221, 109, 16}
\definecolor{themeTeal}{RGB}{18, 54, 69}
\definecolor{themeGrey}{RGB}{120, 121, 124}
\definecolor{themePurple}{RGB}{130, 85, 175}
\definecolor{themeGreen}{RGB}{145, 170, 75}


\newcommand{\VoltageSuffix}{\text{\lVoltageInputSuffix}}  % or empty
\ifdefstring{\UseGenericVoltageLabel}{true}{
    \renewcommand{\VoltageSuffix}{}%
}{
    \renewcommand{\VoltageSuffix}{\text{vst}}%
}

\begin{document}
    \def\bitlist{}
    \begin{tikzpicture}[xscale=1, yscale=.5]
%        \draw[step=0.5, very thin, color=gray!30] (0,0) grid (4.5,8.5);

        % Axes
        \draw[-latex, themeGrey] (0,0) -- (5,0) node[right] {\tiny $\text{\lTime}$};
        \draw[-latex, themeGrey] (0,0) -- (0,8) node[above] {\tiny $\text{\lVoltage}_{\VoltageSuffix}$};

        % Y-axis ticks (8 total: 0–7)
        \ifdefstring{\QuantizationLevelsVisible}{true} {%
            \draw (0, 0) node [left] {\tiny 0};
        }

        \foreach \y in {1,...,7} {
            \draw[themeGrey] (0.1,\y) -- (0,\y);

            \ifdefstring{\QuantizationLevelsVisible}{true} {%
                \draw (0, \y) node [left] {\tiny \y};
            }

        }

        \ifdefstring{\QuantizationLevelsVisible}{true} {%
            \foreach \y in {0,1, ..., 7} {%
                \draw [themeTeal, opacity=.6] (0,\y) -- (4.75, \y);
            }

            \ifdefstring{\QuantizationLevelLabelsVisible}{true} {%
                \draw (4.75, 7) node [anchor=west, themeTeal] {\textsized{4}{\lQuantizationLevel}};
            }

            \ifdefstring{\QuantizationSubLevelsVisible}{true} {%

                \foreach \y in {0.5, 1.5, ..., 7.5} {%
                    \draw [dashed, themeGrey, opacity=.3] (0,\y) -- (4.75, \y);
                }

            }

           \ifdefstring{\QuantizationLevelLabelsVisible}{true} {%

                \draw (4.75, 7.5) node [anchor=west, themeGrey] {\textsized{4}{\lQuantizationDecisionLevel}};

            }
        }
        \typeout{--- START Analog Block ---}
        % Analog waveform
        \ifdefstring{\AnalogSignalVisible}{true} {

            \typeout{AnalogSignalVisible: [\AnalogSignalVisible]}

            \draw[very thick, themeOrange] plot[smooth, tension=.55] coordinates {
                (-0.2,3)    % virtual
                (0, 3.2)    % virtual
                (.25, 3.8)  % virtual
                (0.5,4.78)
                (.76,5.8)   % virtual
                (1.0,6.4)
                (1.2,6.6)   % virtual
                (1.5,6.55)
                (2.0,5.88)
                (2.5,5.48)
                (3.0,5.50)
                (3.5,6.45)
                (4.0,6.585)
                (4.5,5.795)
                (4.75,5.5)  % virtual
            };

        }
        \typeout{--- END Analog Block ---}

        \foreach [count=\i from 1] \ytop in {
            4.78,
            6.4,
            6.55,
            5.88,
            5.48,
            5.50,
            6.45,
            6.585,
            5.795
        } {
            % Calculate quantization points
            \edef\x{\fpeval{0.5 * \i}}
            \edef\yquant{\fpeval{round(\ytop, 0)}}
            \edef\ydiff{\fpeval{\ytop - \yquant}}
            \edef\yvis{\fpeval{0.5 * \yquant}}

            \xdef\bitlist{\bitlist,\yquant}

            \ifdim\ydiff pt>0pt
                \def\pointcolor{themeBlue}
            \else
                \def\pointcolor{themeGreen}
            \fi

            \ifdefstring{\SampleLineVisible}{true} {%
                \draw [ultra thick, themeBlue, opacity=.5] ({0.5 * \i}, \ytop) -- ({0.5 * \i}, 0.01);
            }

            \ifdefstring{\SampleIntervalsVisible}{true} {%
                \draw[dashed, themeGrey, opacity=.5] ({.5*\i},0) -- (.5*\i, \ytop);
            }

            \ifdefstring{\SamplePointsVisible}{true} {%
                \begin{scope}[yscale=2] % reverse scale
                    \fill[themePurple] ({0.5 * \i}, .5*\ytop) circle[radius=1.5pt] coordinate (c top);

                    \ifdefstring{\SamplePointValueVisible}{true} {%
                        \draw (c top) node[above, themePurple] {\tiny \fpeval{round(\ytop, 2)}};
                    }

                \end{scope}
            }


            \begin{scope}[yscale=2] % reverse scale

                \ifdefstring{\QuantizationPointLineVisible}{true} {%

                    \draw [ultra thick, themeGreen, opacity=.5] (\x, 0.01) -- (\x, \yvis);

                }

                \ifdefstring{\QuantizationPointsVisible}{true} {%

                    \fill[\pointcolor] (\x, \yvis) circle[radius=1.5pt] coordinate (c quant);

                    \ifdefstring{\QuantizationPointValueVisible}{true} {%

                        \draw (c quant) node[above, \pointcolor] {\tiny \yquant};

                    }

                }

            \end{scope}

        }

        \ifdefstring{\SamplingImpulsesVisible}{true} {%

            \begin{scope}[yshift=-4cm]
                \draw[-latex, themeGrey] (0,0) -- (5,0) node[right] {\tiny $\text{\lTime}$};
                \draw[-latex, themeGrey] (0,0) -- (0,3) node[above] {\tiny $\text{\lVoltage}_{\text{\lVoltageSamplingSuffix}}$};

                \foreach \x in {0.5, 1, ..., 4.5}{%

                    \draw[ultra thick, themeGreen!40] (\x, 0.012) -- (\x, 2);

                }
            \end{scope}

        }

        \ifdefstring{\CodingVisible}{true} {%
            \ifdefstring{\SamplingImpulsesVisible}{true}{
                \def\myYShift{-7cm}  % More space if analog signal is visible
            }{
                \def\myYShift{-4cm}  % Default shift
            }

            \begin{scope}[yshift=\myYShift]
                \draw[-latex, themeGrey] (0,0) -- (5,0) node[right] {\tiny $\text{t}$};

                \foreach \x [count=\i from 0] in {0.5, 1, ..., 4.5}{%

                    \draw[line width=6, themeBlue!40] (\x, 0.012) -- (\x, 2);

                }

                % Strip leading commas
                \def\stripcomma,#1{#1}
                \edef\bitlist{\expandafter\stripcomma\bitlist}

                \typeout{Bitlist: \bitlist}

                \foreach [count=\i from 1] \xnum in \bitlist {
                    % Convert decimal to binary
                    \pgfmathsetmacro{\bina}{int(mod(\xnum,2))}
                    \pgfmathsetmacro{\temp}{int(\xnum/2)}
                    \pgfmathsetmacro{\binb}{int(mod(\temp,2))}
                    \pgfmathsetmacro{\temp}{int(\temp/2)}
                    \pgfmathsetmacro{\binc}{int(mod(\temp,2))}
                    \pgfmathsetmacro{\temp}{int(\temp/2)}
                    \pgfmathsetmacro{\bind}{int(mod(\temp,2))}

                    % Assemble vertical binary
                    \edef\binarylabel{\binc\\\binb\\\bina}

                    \draw[line width=6, themeBlue!40] (\i*0.5, 0.012) -- (\i*0.5, 2);
                    \node at (\i*.5,1) [
                         align=center,
                         text width=6pt,
                         font=\fontsize{4pt}{4pt}\selectfont
                    ] { \binarylabel };
                }

            \end{scope}

        }

        \ifdefstring{\ReconstructionVisible}{true} {%
            \draw[ultra thick, dashed, themePurple] (0, \pgflinewidth) -- ($ (.5, \pgflinewidth) + (.5*\pgflinewidth, 0) $);

            \foreach [count=\i from 1, remember=\xnum as \lastxnum (initially 0), remember=\i as \lasti (initially 1)] \xnum in \bitlist {

                \draw[ultra thick, themePurple] (.5*\lasti, \lastxnum) |- ($ (.5*\i, \xnum) + (.5*\pgflinewidth, 0) $);

            }

        }

    \end{tikzpicture}
\end{document}
