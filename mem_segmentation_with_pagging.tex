\documentclass[margin=2pt]{standalone}
\usepackage[table]{xcolor}
\usepackage[utf8]{inputenc}
\usepackage[T1]{fontenc}

\usepackage{lmodern}
\usepackage{helvet}

\usepackage{tikz}

\usepackage{amsmath}
\usepackage[czech]{babel}
\usepackage{csquotes}
\usepackage{xparse}

\renewcommand{\familydefault}{\sfdefault}

% Use \phantom to hide text for exams
\renewcommand{\phantom}{}

\iflanguage{czech}{
	\newcommand{\lCPU}{CPU}
	\newcommand{\lLimitValue}{limit}
	\newcommand{\lBaseValue}{báze}
	\newcommand{\lFrame}{rámec}
	\newcommand{\lSegmentPage}{Tabulka \phantom{segmentů}}
	\newcommand{\lMemory}{Fyzická paměť}
	\newcommand{\lAddressLogical}{\phantom{Logická} {adresa}}
	\newcommand{\lAddressPhysical}{\phantom{Fyzická} {adresa}}
	\newcommand{\lAddressLinear}{\phantom{Lineární} {adresa}}
	\newcommand{\lAddressRelative}{\phantom{Relativní} {adresa}}
	\newcommand{\lAddressBase}{\phantom{Bázová }{adresa}}
	\newcommand{\PhysicalMemory}{Fyzická paměť}
	\newcommand{\lNumber}{{číslo}}
	\newcommand{\lPage}{{stránky}}
	\newcommand{\lPageTable}{Tabulka stránek}
	\newcommand{\lLinearProcessMemory}{Lineární paměť\\procesu}
	\newcommand{\lNonLinearProcessMemory}{Nespojitá\\paměť procesu}
	\newcommand{\lFrameNumber}{č. rámce}
	\newcommand{\lOffset}{offset}
	\newcommand{\lPages}{stránky}
	\newcommand{\lSegment}{segment}
	\newcommand{\lSelector}{selektor}
	\newcommand{\lCtrlBits}{ř. bity}
	\newcommand{\lLimit}{limit}
	\newcommand{\lPageDirectory}{str. tab.}
	\newcommand{\lAddressShort}{adr.}
	\newcommand{\lPageShort}{str.}
	\newcommand{\lTableSegment}{Segmentová tabulka}
	\newcommand{\lTablePages}{Stránková tabulka}
	\newcommand{\lTablePageDirectory}{Stránkový adresář}
	\newcommand{\lBits}{bitů}
	\newcommand{\lSegmentation}{segmentace}
	\newcommand{\lPagging}{stránkování}
}

\iflanguage{english}{
	\newcommand{\lCPU}{CPU}
	% ... redefine for rest of the terms in english language or define new language
}

%helper macro
\newcommand{\betweenX}[3]{($(#1.#2 west)!#3!(#1.#2 east)$)}
\newcommand{\betweenXs}[2]{($(#1.west)!#2!(#1.east)$)}


\usetikzlibrary{
	hobby,
	babel,
	intersections,
	spath3,
	shapes.arrows,
	shapes.geometric,
	shapes.symbols,
	fit,
	backgrounds,
	calc,
	tikzmark,
	decorations.pathreplacing,
	angles,
	arrows.meta,
	quotes,
	positioning,
}

\definecolor{themeBlue}{RGB}{1, 103, 143}
\definecolor{themeOrange}{RGB}{221, 109, 16}
\definecolor{themeTeal}{RGB}{18, 54, 69}
\definecolor{themeGrey}{RGB}{120, 121, 124}
\definecolor{themeGreen}{HTML}{99ced3}

\begin{document}
	\begin{tikzpicture}[
		label/.style={font=\scriptsize\bfseries, align=center},
		mem label/.style={font=\scriptsize, align=center},
		mem cell width/.style={minimum width=2.75cm},
		mem cell height/.style={minimum height=1.5cm},
		mem cell half height/.style={minimum height=1cm},
		mem cell/.style={mem cell width, mem cell height, draw}, %=themeGrey
		mem cell continue/.style={mem cell width, mem cell half height, draw, tape, fill, fill=white, tape bend top=none, tape bend height=.5em},
		physical address highlight/.style={mem cell, minimum height=1.3em, font=\scriptsize, draw, fill=themeGreen!60},
		linear address highlight/.style={mem cell, minimum height=1.3em, font=\scriptsize, draw, fill=themeOrange!40},
		segment highlight/.style={mem cell, minimum height=1.3em, font=\scriptsize, draw, fill=yellow!40},
		table cell/.style={anchor=west, draw=themeGrey, font=\scriptsize, minimum height=1.2em, minimum width=2em, xshift=-\pgflinewidth},
		overlay/.style={draw=none, outer sep=0, inner sep=0, font=\scriptsize, black},
		plus/.style = {draw, minimum size=.5em, font=\scriptsize\bfseries, fill=themeOrange!30, node contents={$ + $}},
		leq/.style={draw, diamond, minimum size=.3em, font=\scriptsize\bfseries, fill=red!30, node contents={$ \le $}, scale=.7},
	]
	%\draw[help lines] (-15,-15) grid[step=0.25] (15,2);

	\pgfdeclarelayer{underground}
	\pgfdeclarelayer{background}
	\pgfdeclarelayer{foreground}
	\pgfsetlayers{underground,background,main,foreground}
	
	% Linear process memory
	\begin{scope}[name=linear process memory]
	 	\coordinate (h0) at (0,0);
		\node (linear memory label) at (h0) [label, above] {\lLinearProcessMemory};
		
	  	\foreach \i [count=\n from 1] in {0, ..., 6}  {
	  		\node (m \i) at (h\i) [mem cell, anchor=north, yshift=\pgflinewidth]  {};
			\node at (m \i.north east) [mem label, below, left, anchor=north east] {\i};
			\coordinate (h\n) at (m \i.south);
	 	 }
 	 	\node [mem cell continue, anchor=north, yshift=\pgflinewidth] at (m 6.south) (m 7) {};
	 	\node at (m 7.north east) [mem label, below, left, anchor=north east] {7};
		
		\node (linear address 0) at ($ (m 1.north)!.7!(m 1.south) $) [linear address highlight] {};
		\node (linear address 1) at ($ (m 5.north)!.7!(m 5.south) $) [linear address highlight] {};
	 \end{scope}
	 
	 % Physical memory
	 \begin{scope}[xshift=9cm]
		\node (pm 0) at (0,0) [mem cell continue, anchor=north, yshift=-1.5cm, rotate=-180] {};
		\coordinate (h0) at (pm 0.north);
		
		\node (physical memory label) at (h0|-linear memory label) [label] {\PhysicalMemory};
		
	  	\foreach \i [count=\n from 1] in {0, ..., 5}  {
	  		\node (pm \n) at (h\i) [mem cell, anchor=north, yshift=\pgflinewidth]  {};
			\coordinate (h\n) at (pm \n.south) ;
	 	 }
 	 	\node (pm 7) at (h6) [mem cell continue, anchor=north, yshift=\pgflinewidth]  {};
		
		\node (physical address 0) at ($ (pm 2.north)!.7!(pm 2.south) $)  [physical address highlight] {};
		\node (physical address 1) at ($ (pm 6.north)!.7!(pm 6.south) $)  [physical address highlight] {};
	 \end{scope}
	 
	 % Nonlinear process memory
	 \begin{scope}[xshift=-9cm]
		\coordinate (h0) at (0,0);
		\node (physical memory label) at (h0|-linear memory label) [label] {\lNonLinearProcessMemory};
		
	  	\foreach \i [count=\n from 1] in {0}  {
	  		\node  (segment 0 \n) at (h\i) [mem cell, anchor=north, outer sep=0,yshift=.5\pgflinewidth] {};
			\coordinate (h\n) at (segment 0 \n.south);
	 	 }
		 \node (segment 0 1 label) at (segment 0 1.north west) [below left] {0};
		 
		 \coordinate (h0) at ($ (h1) -(0, .5cm) $);
		 \foreach \i [count=\n from 1] in {0, ..., 2}  {
	  		\node (segment 1 \n) at (h\i) [mem cell, draw=none, anchor=north, yshift=\pgflinewidth, outer sep=0] {};
			\coordinate (h\n) at (segment 1 \n.south);
	 	 }
		 \node (segment 1 wrapper) [draw, fill=themeBlue!10, inner sep=0, outer sep=0, fit=(segment 1 1)(segment 1 2)(segment 1 3)] {};
		 \node (segment 1 1 label) at (segment 1 wrapper.north west) [below left] {1}; 
		 
		 \coordinate (h0) at ($ (h3) -(0, .5cm) $);
		 \foreach \i [count=\n from 1] in {0}  {
	  		\node (segment 2 \n)  at (h\i) [mem cell, draw=none, anchor=north, yshift=\pgflinewidth, outer sep=0] {};
			\coordinate (h\n) at (segment 2 \n.south);
	 	 }
		 \node (segment 2 wrapper) [draw, inner sep=0, outer sep=0, fit=(segment 2 1)] {};
		 \node (segment 2 1 label) at (segment 2 wrapper.north west) [below left] {2}; 
		 
		 \coordinate (h0) at ($ (h1) -(0, .5cm) $);
		 \foreach \i [count=\n from 1] in {0, ..., 1}  {
	  		\node (segment 3 \n) at (h\i) [mem cell, draw=none, anchor=north, yshift=\pgflinewidth, outer sep=0]   {};
			\coordinate (h\n) at (segment 3 \n.south);
	 	 }
		 \node (segment 3 wrapper) [draw, fill=themeBlue!10, inner sep=0, outer sep=0, fit=(segment 3 1)(segment 3 2)] {};
		 \node (segment 3 1 label) at (segment 3 wrapper.north west) [below left] {3}; 
		
		\node (focused segment 0) at ($ (segment 1 2.north)!.7!(segment 1 2.south) $) [segment highlight] {};
		\node (focused segment 1) at ($ (segment 3 2.north)!.7!(segment 3 2.south) $) [segment highlight] {};
	 \end{scope}
	 
	 	 
	 % Pagging
	 \begin{scope}[shift={($ (m 3.east|-m 0.north)!.5!(pm 3.west|-m 0.north) + (-.5, 0cm) $)}]
	 	% Linear address
		\node (addr) [table cell, anchor=north] {};
		\node (page) at (addr.east) [table cell] {};
		\node (offset) at (page.east) [table cell, minimum width=2.5em] {};
		% Frame + Offset
		\draw (addr.south west) ++(down:2em) node (table page) [anchor=north west, table cell, minimum width=4em, xshift=\pgflinewidth] {};
		\node (table offset) at (table page.east) [table cell, minimum width=2.5em]  {};
		% Pages table
		\path (table page.south west) ++(down:6em) coordinate (h0);
		\foreach \i [count=\n from 1] in {0, ..., 5}  {
			\node (table page \i) at (h\i) [table cell, anchor=north west, minimum width=4em, xshift=\pgflinewidth, yshift=\pgflinewidth]  {};
			\node (table offset \i) at (table page \i.east)[table cell, minimum width=2.5em] {};
			\node (table page \i\space label) at  (table page \i.west) [left, font=\scriptsize] {\i};
			\coordinate (h\n) at (table page \i.south west) ;
		}
		% Pages directory
		\path (table page 5.south west) ++(down:2.5em) coordinate (h0);
		\foreach \i [count=\n from 1] in {0, ..., 1}  {
			\node (pages \i) at (h\i) [table cell, anchor=north west, minimum width=4em, xshift=\pgflinewidth, yshift=\pgflinewidth]  {};
			\node (offsets \i) at (pages \i.east) [table cell, minimum width=2.5em] {};
			\node (pages \i\space label) at (pages \i.west) [left, font=\scriptsize] {\i};
			\coordinate (h\n) at (pages \i.south west);
		}
		% Example Frame + offset
		\node (x 0) at (pages 1.south west|-physical address 1.west) [anchor=north west, table cell, minimum width=4em, xshift=\pgflinewidth]  {};
		\node (y 0) at (x 0.east) [table cell, minimum width=2.5em]  {};
		% Example linear address
		\draw (x 0.south west) ++(down:2em) node (addr 2) [anchor=north west, table cell, minimum width=2em, xshift=\pgflinewidth] {};
		\node (page 2) at (addr 2.east) [table cell]  {};
		\node (offset 2) at (page 2.east) [table cell, minimum width=2.5em] {};
	 \end{scope}
	 
	 % NPM <--> LPM
	 \begin{scope}[shift={($ (m 0.north|-segment 2 1.east)!.5!(segment 2 1|-m 0.north) + (-1, 0cm) $)}]
	 	\coordinate (o) at (0,0);
	 	\node (relative address selector) at (o|-m 0.north) [table cell, minimum width=3em, anchor=north]  {};
		\node (relative address offset) at ($ (relative address selector.east) + (.5em, 0) $) [table cell, minimum width=6em]  {};
		
		\draw ($ (relative address selector.south west)!.5!(relative address offset.south east) $) ++(down:3.5cm) coordinate (h0);
		
		\foreach \i in {0, ..., 3} {
			\node (segment table \i\space base) at (h0) [table cell, minimum width=8em, anchor=north, yshift=\pgflinewidth, xshift=\pgflinewidth]  {};
			\node (segment table \i\space ctrl) at (segment table \i\space base.south west) [table cell, anchor=north west, minimum width=3em, yshift=\pgflinewidth, xshift=\pgflinewidth] {};
			\node (segment table \i\space limit) at (segment table \i\space ctrl.east) [table cell, minimum width=5em, anchor=west] {};
			\node (segment table \i\space label) at (segment table \i\space base.south west) [left, mem label] {\i};
			\coordinate (h0) at ($ (segment table \i\space ctrl.south west)!.5!(segment table \i\space limit.south east) $);
		}
		
		\node (example relative address selector) at (o|-m 7.north) [table cell, minimum width=3em, anchor=north]  {};
		\node (example relative address offset) at ($ (example relative address selector.east) + (.5em, 0) $) [table cell, minimum width=6em]  {};
		
		\node (plus 0) at ($ (linear address 0.west) -(2.5em, 0) $) [plus];
		\node (plus 1) at ($ (linear address 1.west) -(2.5em, 0) $) [plus];
		\node (check) at ($ (plus 1.west-|relative address selector.east) +(right:1.25em) $) [leq];
	\end{scope}
	
	% Internal labels & arrows
	\begin{scope}[latex-latex, font=\scriptsize]
		\draw[themeGreen!70!black] 
			(pm 2.north) 
				-- node[midway, right] {\lOffset} 
			(physical address 0.north);
		\draw[red!60!themeGreen] 
			(pm 6.north) 
				-- node[midway, right] { \texttt{0xA48} } (
			physical address 1.north);
			
		\draw[themeOrange!70!black]  
			\betweenX{focused segment 0}{north}{.2} 
				-- node[midway, right] {\lOffset} 
			\betweenX{segment 1 1}{north}{.2};
		\draw[themeOrange!70!black]  
			\betweenX{focused segment 1}{north}{.2} 
				-- node[midway, right] { \texttt{0x1A48} } 
			\betweenX{segment 3 1}{north}{.2};
		
		\draw[themeOrange!70!black]  
			\betweenX{linear address 1}{north}{.2} 
				-- node[midway, right] {\texttt{0x1A48}}  
			\betweenX{m 4}{north}{.2};
		\draw[themeGreen!70!black]
			\betweenX{linear address 1}{north}{.65} 
				-- node[midway, right] { \texttt{0xA48} }
			\betweenX{m 5}{north}{.65};
		
		\draw[themeOrange!70!black] 
			\betweenX{linear address 0}{north}{.2} 
				-- node[midway, right] {\lOffset}
			\betweenX{m 0}{north}{.2};
			
		\draw[themeGreen!70!black]
			\betweenX{linear address 0}{north}{.65} 
				-- node[midway, right] {\lOffset}
			\betweenX{m 1}{north}{.65};
	\end{scope}
	
	% Overlay labels & background fills 
	
	\begin{scope}[overlay, name=labels]
		\begin{pgfonlayer}{foreground} % send to background by changing the layer to underground or background
			\node at (segment 0 1) {\lSegment};
			\node at (offset.center) {\lOffset};
			\node at (table offset)  {\lOffset};
			
			\node at (addr) {\lAddressShort};
			\node at (page) {\lPageShort};
			
			\node (segment table label) at ($ (segment table 0 base.north) +(0, .4em) $) [above] {\lTableSegment};
			\node (selector-offset label) at ($ (relative address selector.north west)!.5!(relative address offset.north east) +(0, .4em) $) [above] {\lAddressLogical};
		
			\node (segment table 0 base label) at (segment table 0 base.center) [] {32 \lBits};
			\node (segment table 0 ctrl label) at (segment table 0 ctrl.center) [] {12 b};
			\node (segment table 0 limit label) at (segment table 0 limit.center) [] {20 \lBits};

			\node (segment table 1 base label) at (segment table 1 base.center) { \lAddressBase };
			\node (segment table 1 ctrl label) at (segment table 1 ctrl.center) {\lCtrlBits};
			\node (segment table 1 limit label) at (segment table 1 limit.center) {\lLimit};
	
			\node (segment table 3 base label) at (segment table 3 base.center) [] {\texttt{0x00004000}};
			%\node (segment table 3 ctrl label) at (segment table 3 ctrl.center) [] {};
			\node (segment table 3 limit label) at (segment table 3 limit.center) [] {\texttt{0x01FFF}};
			
			\node at (example relative address selector) { \texttt{0x0003} };
			\node at (example relative address offset) { \texttt{0x00001A48} };
			\node at (relative address selector) {\lSelector};
			
			\node at (focused segment 0) {\lAddressRelative};
			\node at (focused segment 1) { \textbf{\texttt{0x00001A48}} };
			
			\node at (table page) {\lFrameNumber};
			
			\node at (physical address 0) {\lAddressPhysical};
			\node at (physical address 1) { \textbf{\texttt{0x4F8E2\textcolor{red!60!themeGreen}{A48}}} };
			\node at (relative address offset) {\lOffset};
			
			\node at (addr 2) { \texttt{0x0} };
			\node at (page 2) { \texttt{005} };
			\node at (offset 2) { \texttt{A48} };
			
			\node at (linear address 0) {\lAddressLinear};
			\node at (linear address 1) {\textbf{\texttt{0x00005A48}} };
			
			\node (leq accounting 0) at ($ (check.north) +(0, .2em) $) [anchor=south, minimum  width=1.5em, align=center]{\texttt{0x1A48} \\ $\le$ \\ \texttt{0x1FFF} };
			\node at ($ (check.south) -(0, .5em) $) { \textbf{OK} };

			\node (plus 1 accounting 0) at ($ (plus 1.west) +(-1.5em, .6em) $) { \texttt{0x4000} };
			\node (plus 1 accounting 1) at ($ (plus 1 accounting 0.south east) -(0, .25em) $) [anchor=north east] { \texttt{+0x1A48} };
			\node (plus 1 accounting 2) at ($ (plus 1 accounting 1.south east) -(0, .25em) $) [anchor=north east] { \texttt{=0x5A48} };
			
			\node at (pm 1) [] {\lFrame};
			\node at (pm 6.north west) [left, align=right] { \texttt{0x4F8E2} };
			
			\node (t 5 frame) at (x 0.center) { \texttt{0x4F8E2} };
			\node (t 5 offset) at (y 0.center) { \texttt{A48} };

			\node at (table page 1) {\lFrameNumber};
			\node at (table offset 1) {\lCtrlBits};
			\node at (table page 5) { \texttt{0x4F8E2} };
					
			\node at (pages 0) [] {\lPageDirectory};
			\node at (m 3) [mem label, text width=4em] {4KiB\\\lPages};
					
			\node (lin addr label) at ($ (addr.north west)!.5!(offset.north east) +(0, .4em) $) [above] {\lAddressLinear};
			\node (table page label) at ($ (table page 0.north west)!.5!(table offset 0.north east) +(0, .4em) $) [above] {\lTablePages};
			\node (pages label) at ($ (pages 0.north west)!.5!(offsets 0.north east) +(0, .4em) $) [above] {\lTablePageDirectory};
		\end{pgfonlayer}
	\end{scope}

	\begin{scope}[on background layer, overlay, name=colored backgrounds]
		\foreach \e in {m 3, m 6, m 7, pm 0, pm 1, pm 3, pm 4, pm 5, pm 7, segment 0 1, segment 2 wrapper, offset, y 0, offset 2, table offset} {
			\node [fit=(\e), fill=white] {};
		}			
		\foreach \i in {0, ..., 3} {
			\node [fit=(segment table \i\space base)(segment table \i\space ctrl)(segment table \i\space limit), fill=white] {};
		}
		
		\foreach \i in {0, ..., 5} {
			\node [fit=(table page \i)(table offset \i), fill=white] {};
		}
		\foreach \i in {0, ..., 1} {
			\node [fit=(pages \i)(offsets \i), fill=white] {};
		}
		
		\node [fit=(table page), fill=themeGreen!20] {};
		
		\node [fit=(pm 2), fill=themeGreen!20] {};
		\node [fit=(pm 6), fill=themeGreen!20] {};
				
		\node [fit=(table page 1)(table offset 1), fill=themeGreen!20] {};
		\node [fit=(table page 5)(table offset 5), fill=themeGreen!20] {};
		\node [fit=(table offset 1), fill=themeGreen!40] {};
		\node [fit=(table offset 5), fill=themeGreen!40] {};
				
		\node [fit=(addr), fill=purple!20] {};
		\node [fit=(pages 0)(offsets 0), fill=purple!20] {};
		\node [fit=(offsets 0), fill=purple!40] {};
				
		\node [fit=(addr 2), fill=purple!20] {};
		\node [fit=(page 2), fill=green!10] {};
				
		\node [fit=(page), fill=green!10] {};
		\node [fit=(x 0), fill=themeGreen!20] {};
		
		\node [fit=(relative address selector), fill=themeOrange!40] {};
		\node [fit=(relative address offset), fill=yellow!30] {};
		
		\node [fit=(example relative address selector), fill=themeOrange!40] {};
		\node [fit=(example relative address offset), fill=yellow!30] {};

		\node [fit=(segment table 1 base), fill=pink!30] {};
		\node [fit=(segment table 1 ctrl), fill=pink!60] {};
		\node [fit=(segment table 1 limit), fill=pink!90] {};

		\node [fit=(segment table 3 base), fill=pink!30] {};
		\node [fit=(segment table 3 ctrl), fill=pink!60] {};
		\node [fit=(segment table 3 limit), fill=pink!90] {};
		
		\node [fit=(m 0)(m 1)(m 2), fill=themeBlue!10] {}; % Segments highlight
		\node [fit=(m 4)(m 5), fill=themeBlue!10] {};

		\node [thick, draw=themeOrange!90!black, fit=(addr)(page)(offset)] {}; % Address highlight
		\node [thick, draw=themeOrange!90!black, fit=(addr 2)(page 2)(offset 2)] {};
		\node [thick, draw=themeGreen, fit=(x 0)(y 0)] {};
		\node [thick, draw=themeGreen, fit=(table offset)(table page)] {};
	\end{scope}
	 
	 % Arrows
	 \begin{scope}[-latex, thick,  rounded corners=4pt, name=arrows]
	 	\draw [themeBlue, |{Triangle}-{Triangle}|] ($ (m 0.north west) - (0.8em, 0) $) -- ($ (m 2.south west) - (0.8em, 0) $);
		\draw [themeBlue, |{Triangle}-{Triangle}|] ($ (m 4.north west) - (0.8em, 0) $) -- ($ (m 5.south west) - (0.8em, 0) $);
		\draw [themeBlue] (segment table 3 limit.east) -- ($ (segment table 3 limit.east-|m 4.west) -(0.8em, 0) $);
		\draw [themeBlue] (segment table 1 limit.east) -- ++(1,0) to (\tikztostart|-m 1.south west) -- ($ (m 1.south west) -(.8em, 0)  $);


		\draw [red] (relative address selector.south) -- ++(0, -.5) -- ++(-1, 0) to (\tikztostart|-segment table 1 label.west) -- (segment table 1 label.west);
		\draw [red] (segment 1 wrapper.north east) -- ++(.75, 0) to (\tikztostart|-relative address selector.west) -- (relative address selector.west);
		\draw [red] (example relative address selector.north) -- ++(0, .5) -- ++(-1, 0) to (\tikztostart|-segment table 3 label.west) -- (segment table 3 label.west);
		\draw [red] (segment 3 wrapper.north east) -- ++(.75, 0) to (\tikztostart|-example relative address selector.west) -- (example relative address selector.west);
	 
	 	\draw [themeOrange!90!black] (linear address 0.east) to ++ (.5, 0) to (\tikztostart|-addr.west) -- (addr.west);
		\draw [themeOrange!90!black] (linear address 1.east) to ++ (.5, 0) to (\tikztostart|-addr 2.west) -- (addr 2.west);
		
		\draw [themeTeal] (table offset.east) -- ++(.75, 0) |- (physical address 0.west);
	 
		\draw [red!70!black] (offsets 0) -- ++(.9,0) to(\tikztostart|-table offset 0.north east) to (table offset 0.north east);
		
		\draw (segment table 1 base.east) -- (segment table 1 base.east-|plus 0.south) to (plus 0.south);
		\draw (relative address offset.south) -- ++(0, -.5) to (\tikztostart-|plus 0.north) -- (plus 0.north);
		
		\draw[green!70!black] (page.south) |- ++ (-1.75, -.35)  |-  (table page 1 label.west);
		\draw[red!70!black] (addr.south) |- ++(-1.25, -.15) |- (pages 0 label);
		
		\draw[green!70!black] (page 2.north) |- ++ (-1.75, .35)  |-  (table page 5 label.west);
		\draw[red!70!black] (addr 2.north) |- ++(-1.25, .15) |- (pages 0 label);
		
		\draw (plus 0.east) -- (linear address 0.west);
		\draw (plus 1.east) -- (linear address 1.west);
		\draw (y 0.east) -- (physical address 1.west);
			
		\draw [] (focused segment 0.east) -- ++(right:.75) -- ++(up:2) to (\tikztostart-|relative address offset.south west) to (relative address offset.south west);
		
		\draw (example relative address offset.north) -- ++(up:.5) to (\tikztostart-|plus 1.south) -- (plus 1.south);
		\draw (focused segment 1.east) to (focused segment 1.east-|example relative address offset.south) -- (example relative address offset.south);
		\draw (segment table 3 limit.south) to (\tikztostart|-check.east) -- (check.east);
		\draw (example relative address offset.north west) -- ++(0, .5) -- ++(-.3, 0) to (\tikztostart|-check.west) -- (check.west);
		
		\draw (segment table 3 base.east) to (\tikztostart-|plus 1.north) to (plus 1.north);
		\draw (segment table 3 base.east) to (\tikztostart-|plus 1.north) to (\tikztostart|-m 4.north west) -- (m 4.north west);
		
		\draw (offset) -- (table offset);
		\draw (offset 2) -- (y 0);
		
		\draw [themeGreen!50!black] (table page 1.north) |- ++ (1.9, .2) -- ++(0, .8) to (table page.south|-\tikztostart) -- (table page.south);
		\draw [themeGreen!50!black] (table page 5.south) |- ++ (1.9, -.2) -- ++(0, -1.8) to (x 0.north|-\tikztostart) -- (x 0.north);
	 \end{scope}
	 
	 \begin{scope}[overlay, name=frames]	
		\node [thick, draw=themeOrange!90!black, fit=(addr)(page)(offset)] {};
		\node [thick, draw=themeOrange!90!black, fit=(addr 2)(page 2)(offset 2)] {};
		\node [thick, draw=themeGreen, fit=(x 0)(y 0)] {};
		\node [thick, draw=themeGreen, fit=(table page)(table offset)] {};
	\end{scope}
	
	\begin{scope}[name=backdrop]
	 	\begin{pgfonlayer}{underground}
			\node (segmentation) [rounded corners=4pt, densely dashed, thick, fill=themeOrange!5, inner sep=3em, fit=(m 0.west) (m 7.west) (segment 0 1) (segment 3 wrapper)] {};
			\node (pagging) [rounded corners=4pt, densely dashed, thick, fill=themeGreen!10, inner sep=3em, fit=(m 0.east) (m 7.east) (pm 0.east|-segment 0 1.north) (pm 7.east|-segment 3 wrapper.south)] {};
			\node at (segmentation.south) [above, font=\bfseries] { \lSegmentation };
			\node at (pagging.south) [above, font=\bfseries] {stránkování};
		\end{pgfonlayer}
	\end{scope}
		
	\end{tikzpicture}
\end{document}
