% ==============================================================================
% @file        adc_ideal.tex
% @author      Tomáš Michalek
% @date        2025-06-28
% @version     1.0
%
% @copyright   Copyright (C) 2025 SPŠE Havířov
% @license     EUPL-1.2-or-later
% @SPDX-License-Identifier EUPL-1.2
%
% @description
% This TikZ diagram illustrates the transfer characteristic of an ideal
% Analog-to-Digital Converter (ADC) with a 3-bit resolution.
%
% The following elements are included:
% - Continuous (orange) reference line representing an ADC with infinite resolution
% - Stepwise (blue) quantization characteristic of a 3-bit ideal ADC
% - Bitwise binary output labels aligned with quantization levels
% - Full-scale marker and axis labels for analog input and digital output
% - LSB (Least Significant Bit) visualized both vertically and horizontally
%
% The diagram is fully localized using babel's \iflanguage construct. Czech labels
% are currently provided. To support more languages, extend the \iflanguage block.
%
% This diagram is designed for teaching digital electronics and signal processing
% in technical secondary schools. It highlights quantization effects and resolution.
%
% @build
% To compile this file:
% > pdflatex adc_ideal.tex
%
% This standalone LaTeX file produces a cropped PDF suitable for use in
% presentations, worksheets, and other educational materials.
%
% @dependencies
% - TikZ and PGF libraries: calc, arrows.meta, shapes.arrows, backgrounds, spy
% - Babel with Czech localization
% - Font and math support: helvet, amsmath, inputenc, fontenc
% - etoolbox, xstring, xparse for preprocessing and logic
% ==============================================================================
\documentclass[tikz, margin=2pt]{standalone}
\usepackage[utf8]{inputenc}
\usepackage[T1]{fontenc}

\usepackage{tikz}
\usepackage{helvet}
\usepackage{amsmath}
\usepackage[czech]{babel}
\usepackage{csquotes}
\usepackage{xparse}
\usepackage{graphicx}
\usepackage{etoolbox}
\usepackage{xstring}

\renewcommand{\familydefault}{\sfdefault}

\iflanguage{czech}{
    \newcommand{\lFullScale}{FS}
    \newcommand{\lLSB}{LSB}
    \newcommand{\lIdealChar}{ideální převodní\\charakteristika}
    \newcommand{\lInfiniteResolution}{charakteristika s\\nekonečným rozlišením}
    \newcommand{\lOutputBinaryCode}{výstupní binární kód}
    \newcommand{\lInputAnalogSignal}{vstupní analogový signál}
}

\usetikzlibrary{
    babel,
    calc,
    arrows.meta,
    shapes.arrows,
    positioning,
    backgrounds,
    fit,
    spy,
}

\pgfdeclarelayer{background}
\pgfsetlayers{background,main}

\pdfsuppresswarningpagegroup=1 % Suppress PDF includes group warnings

\newcommand{\AnalogSignalVisible}{true}
\newcommand{\UseGenericVoltageLabel}{~true}

\makeatletter
% Helper macro to hold node content before uppercasing
\newcommand{\tikz@text@content}{}
\tikzset{
    execute at begin node/.style={
        execute at begin node=$\begingroup\let\tikz@text@content=,
        execute at end node=\endgroup#1
    }
}
\makeatother

\definecolor{themeBlue}{RGB}{1, 103, 143}
\definecolor{themeOrange}{RGB}{221, 109, 16}
\definecolor{themeTeal}{RGB}{18, 54, 69}
\definecolor{themeGrey}{RGB}{120, 121, 124}
\definecolor{themePurple}{RGB}{130, 85, 175}
\definecolor{themeGreen}{RGB}{145, 170, 75}

\newcommand{\scripted}[2]{%
    % #1 = font size (e.g., 12pt)
    % #2 = text to uppercase
    {\fontsize{#1}{#1}\selectfont\strut\MakeUppercase{#2}}%
}

\begin{document}
    \begin{tikzpicture}[%
%        spy using outlines={circle, themeGreen, magnification=1.75, size=3cm, connect spies},
        >>/.style={-{Triangle[length=4pt, width=4pt]}},
        <<>>/.style={{Triangle[length=2pt, width=2pt]}-{Triangle[length=2pt, width=2pt]}},
        axis/.style={>>, thick, themeGrey},
        axis label help/.style={thick, themeGrey},
        fence/.style={themeGrey, dashed},
        lsb helper/.style={themeGrey, dotted},
        lsb arrow/.style={themeGrey, <<>>},
        ]

%        \draw[step=1cm,gray,very thin, opacity=.2] (0,0) grid (9,9);
        \draw [axis] (0, -.5*\pgflinewidth) -- (0, 8);
        \draw [axis] (0, 0) -- (9, 0);

        \draw [fence] (7, 0) -- (7, 7);
        \draw [fence] (0, 7) -- (7, 7);

        \draw [axis label help] (-.1, 0) node [left] { 000 };
        \draw [axis label help] (8, .1) -- (8, -.1) node [below] { \lFullScale };

        \draw [lsb helper] (3.5, 3) -- ++(right:1.5) coordinate (lsb v bot);
        \draw [lsb helper] (4.5, 4) -- ++(right:0.5) coordinate (lsb v top);
        \draw [lsb arrow]  (lsb v top) -- (lsb v bot);
        \draw ($ (lsb v top)!.5!(lsb v bot) $) node[right, themeGrey, ] {\fontsize{5}{5}\selectfont LSB};

        \draw [lsb helper] (3.5, 3) -- ++(down:.5) coordinate (lsb h left);
        \draw [lsb helper] (4.5, 4) -- ++(down:1.5) coordinate (lsb h right);
        \draw [lsb arrow]  (lsb h left) -- (lsb h right);

        \draw ($ (lsb h left)!.5!(lsb h right) $) node [below, themeGrey] {\fontsize{5}{5}\selectfont \lLSB};

        \foreach \y in {0, ..., 7}{%
             % Convert decimal to binary
            \pgfmathsetmacro{\bina}{int(mod(\y,2))}
            \pgfmathsetmacro{\temp}{int(\y/2)}
            \pgfmathsetmacro{\binb}{int(mod(\temp,2))}
            \pgfmathsetmacro{\temp}{int(\temp/2)}
            \pgfmathsetmacro{\binc}{int(mod(\temp,2))}
            \pgfmathsetmacro{\temp}{int(\temp/2)}
            \pgfmathsetmacro{\bind}{int(mod(\temp,2))}
            % Assemble vertical binary
            \edef\binarylabel{\binc\\\binb\\\bina}

            \draw[thick, themeGrey] (.1, \y) -- (-.1, \y) node [left] { \binarylabel };
        }

        \foreach \x in {0, ..., 7}{%
            \draw[thick, themeGrey] (\x, .1) -- (\x, -.1) node [below] {$ \frac{\text{\x}}{\text{8}} $};
        }

        \draw [themeOrange, thick] (0, 0) -- (7.5, 7.5);

        \draw [ultra thick, themeBlue] (0, 0) -- ($ (.5, 0) + (.5*\pgflinewidth, 0)$);
        \foreach [count=\x from 1] \y in {0, ..., 6}{%

            \draw[ultra thick, themeBlue] (\x-.5, \y) |- (\x, \y+1) -- ($ (\x+.5, \y+1) + (.5*\pgflinewidth, 0) $);
        }

        \draw (2, 5) node [themeBlue, align=center] {\lIdealChar};
        \draw (5, 1.5) node [themeOrange, align=center] {\lInfiniteResolution};

        \draw ($ (-1.25, 0)!.5!(-1.25, 8) $) node [themeGrey, rotate=90, align=center] {\fontsize{7}{7}\selectfont \lOutputBinaryCode};
        \draw ($ (0, -1.25)!.5!(8, -1.25) $) node [themeGrey, align=center] {\fontsize{7}{7}\selectfont \lInputAnalogSignal};

%        \spy on (4, 3.5) in node (zoom) [left] at (11, 4);
    \end{tikzpicture}
\end{document}